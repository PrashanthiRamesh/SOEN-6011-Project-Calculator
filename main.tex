\documentclass[11pt, english]{report}
\usepackage{graphicx}
\usepackage[colorlinks=true, linkcolor=blue]{hyperref}
\usepackage[english]{babel}
\selectlanguage{english}
\usepackage[utf8]{inputenc}
\usepackage[svgnames]{xcolor}
\usepackage{url}
\usepackage{hyperref}
\usepackage{float}
\usepackage{longtable}
\usepackage[toc]{glossaries}
\usepackage{amsmath}
\usepackage{amssymb}

\usepackage{algpseudocode}
\usepackage{algorithm}
\usepackage{csquotes}

\newcommand{\R}{\mathbb{R}}


\usepackage{listings}
\usepackage{afterpage}
\pagestyle{plain}

\definecolor{dkgreen}{rgb}{0,0.6,0}
\definecolor{gray}{rgb}{0.5,0.5,0.5}
\definecolor{mauve}{rgb}{0.58,0,0.82}
\usepackage{biblatex}
\bibliography{ref.bib}

%\lstset{language=R,
%    basicstyle=\small\ttfamily,
%   stringstyle=\color{DarkGreen},
%    otherkeywords={0,1,2,3,4,5,6,7,8,9},
%    morekeywords={TRUE,FALSE},
%    deletekeywords={data,frame,length,as,character},
%    keywordstyle=\color{blue},
%    commentstyle=\color{DarkGreen},
%}

\lstset{frame=tb,
language=R,
aboveskip=3mm,
belowskip=3mm,
showstringspaces=false,
columns=flexible,
numbers=none,
keywordstyle=\color{blue},
numberstyle=\tiny\color{gray},
commentstyle=\color{dkgreen},
stringstyle=\color{mauve},
breaklines=true,
breakatwhitespace=true,
tabsize=3
}

\usepackage{here}


\textheight=21cm
\textwidth=17cm
%\topmargin=-1cm
\oddsidemargin=0cm
\parindent=0mm
\pagestyle{plain}

\usepackage{color}
\usepackage{ragged2e}

\global\let\date\relax
\newcounter{unomenos}
\setcounter{unomenos}{\number\year}
\addtocounter{unomenos}{-1}
\stepcounter{unomenos}
\gdef\@date{ Course \arabic{unomenos}/ 2019}

\makeglossaries
 
\newglossaryentry{eternity}
{
    name=ETERNITY: FUNCTIONS,
    description={It stands for the name of both the project and the product, unless otherwise stated}
}

\begin{document}

\begin{titlepage}

\begin{center}
\vspace*{-1in}
\begin{figure}[htb]
\begin{center}
\includegraphics[width=8cm]{logo}
\end{center}
\end{figure}
\begin{Large}
\textbf{SOEN 6011 - Software Engineering Processes} \\
\end{Large}
\vspace*{0.1in}
Summer 2019\\
\vspace*{0.5in}
\begin{Large}
\textbf{Scientific Calculator-  ETERNITY: FUNCTIONS} \\
\end{Large}
\vspace*{0.4in}
\begin{large}
Project Report\\
\end{large}
\vspace*{0.2in}
\begin{Large}
\textbf{Deliverable 1} \\
\end{Large}
\vspace*{0.3in}
\begin{large}
Presented to \\
\vspace*{0.1in}
Instructor: PANKAJ KAMTHAN 
 \\
\end{large}
\vspace*{0.3in}
\rule{80mm}{0.1mm}\\
\vspace*{0.1in}
\begin{large}
By \\
Prashanthi Ramesh\\ 
\vspace*{0.3in}
\date{\normalsize\today} 

\end{large}
\end{center}
\end{titlepage}

\newcommand{\CC}{C\nolinebreak\hspace{-.05em}\raisebox{.4ex}{\tiny\bf +}\nolinebreak\hspace{-.10em}\raisebox{.4ex}{\tiny\bf +}}
\def\CC{{C\nolinebreak[4]\hspace{-.05em}\raisebox{.4ex}{\tiny\bf ++}}}

\tableofcontents
\newpage
\chapter{Problem 1}

\section{Function Definition}

Exponentiation\cite{wiki} is a mathematical operation, written as \(x^y\), involving two numbers, the base x and the exponent or power y. When y is a positive integer, exponentiation corresponds to repeated multiplication of the base: that is, \(x^y\) is the product of multiplying y bases:

\begin{equation} \label{eq1}
f(x,y)= x^y
\end{equation}
 
\ref{eq1} is known as the Exponentiation function

\begin{equation} \label{eq2}
x^y= x*...*x \textrm{ (y times)}
\end{equation}
 
\ref{eq2} is depicts the evaluation of exponentiation function

\section{Domain}

A set of real numbers\cite{wolframalpha}  (-\(\infty\) to \(\infty\))

\begin{equation} \label{eq3}
\{(x,y) \in \R^2 : (x \geq 0 \land y \neq 0) \lor x>0\}
\end{equation}
 
\ref{eq3} presents the domain of exponentiation function

\section{Co-domain}

A set of all positive real numbers\cite{mathbits} (0 to \(\infty\))

\section{Characteristics}

\begin{itemize}
    \item \textbf{Graph: }
    \begin{itemize}
        \item In an exponential graph, the ``rate of change'' increases (or decreases) across the graph.
        \item The exponential graph crosses the y-axis at (0,1).
        \item The exponential graph increases, when x \(>\) 1.
        \item The exponential graph decreases, when 0 \(<\) x \(<\) 1.
        \item The exponential graph is asymptotic to the x-axis - gets very, very close to the x-axis but, in this case, does not touch it or cross it.
    \end{itemize}
    \item \textbf{Injectivity:} \(x^y\) function is injective which means it is one-to-one function
    \item \textbf{Surjective:} \(x^y\) function is not surjective which means it is not onto function
    \item \(x^y=1\) when y = 0  and x \(\neq\) 0 
    \item \(x^y= Undefined\) when y = 0 and x = 0 
    \item \(x^y= Undefined\) when x = 0 and y \(<\) 0  
\end{itemize}

\chapter{Problem 2}

This section presents the assumptions, requirements for implementing the function f(x,y)=\(x^y\).

\section{Assumptions}

\begin{enumerate}
    \item The Eternity Function will accept only real numbers as input
    \item The Eternity Function will be able to handle entry value of double-precision floating-point
    \item The Eternity Function result value will be accurate to fifteen decimal places
\end{enumerate}

\section{Requirements}

\setlength{\tabcolsep}{18pt}
\renewcommand{\arraystretch}{1.5}
\begin{tabular}{ |p{2.2cm}|p{9.8cm}| }
\hline
\textbf{Identification} &  EF\_REQ\_1 \\ \hline 
\textbf{Version Number} & 1.0 \\ \hline 
\textbf{Owner} & Prashanthi Ramesh  \\ \hline
\textbf{Priority} & High  \\ \hline
\textbf{Risk} & Medium  \\ \hline
\textbf{Description} & When the user inputs a value for x as any real number except zero and for y as zero (0), the Eternity Function shall display the result as 1\\ \hline
\textbf{Difficulty} & Nominal  \\ \hline
\textbf{Type} & Functional \\ 
\hline
\end{tabular} \\ \\ \\

\setlength{\tabcolsep}{18pt}
\renewcommand{\arraystretch}{1.5}
\begin{tabular}{ |p{2.2cm}|p{9.8cm}| }
\hline
\textbf{Identification} &  EF\_REQ\_2 \\ \hline 
\textbf{Version Number} & 1.0 \\ \hline 
\textbf{Owner} & Prashanthi Ramesh  \\ \hline
\textbf{Priority} & High  \\ \hline
\textbf{Risk} & Medium  \\ \hline
\textbf{Description} & When the user inputs a value for x as any real number except zero and for y as 1, the Eternity Function shall display the result as the same value as x\\ \hline
\textbf{Difficulty} & Nominal  \\ \hline
\textbf{Type} & Functional \\ 
\hline
\end{tabular}  \\ \\ \\ 

\setlength{\tabcolsep}{18pt}
\renewcommand{\arraystretch}{1.5}
\begin{tabular}{ |p{2.2cm}|p{9.8cm}| }
\hline
\textbf{Identification} &  EF\_REQ\_3 \\ \hline 
\textbf{Version Number} & 1.0 \\ \hline 
\textbf{Owner} & Prashanthi Ramesh  \\ \hline
\textbf{Priority} & High  \\ \hline
\textbf{Risk} & Medium  \\ \hline
\textbf{Description} & When the user inputs a value for x as any real number and for y as a unreal, non-numeric value, the Eternity Function shall display an error message\\ \hline
\textbf{Difficulty} & Nominal  \\ \hline
\textbf{Type} & Functional \\ 
\hline
\end{tabular} \\ \\ \\

\setlength{\tabcolsep}{18pt}
\renewcommand{\arraystretch}{1.5}
\begin{tabular}{ |p{2.2cm}|p{9.8cm}| }
\hline
\textbf{Identification} &  EF\_REQ\_4 \\ \hline 
\textbf{Version Number} & 1.0 \\ \hline 
\textbf{Owner} & Prashanthi Ramesh  \\ \hline
\textbf{Priority} & High  \\ \hline
\textbf{Risk} & Medium  \\ \hline
\textbf{Description} & When the user inputs a value for x a unreal, non-numeric value and for y as a real number, the Eternity Function shall display an error message\\ \hline
\textbf{Difficulty} & Nominal  \\ \hline
\textbf{Type} & Functional \\ 
\hline
\end{tabular} \\ \\ \\

\setlength{\tabcolsep}{18pt}
\renewcommand{\arraystretch}{1.5}
\begin{tabular}{ |p{2.2cm}|p{9.8cm}| }
\hline
\textbf{Identification} &  EF\_REQ\_5 \\ \hline 
\textbf{Version Number} & 1.0 \\ \hline 
\textbf{Owner} & Prashanthi Ramesh  \\ \hline
\textbf{Priority} & High  \\ \hline
\textbf{Risk} & Medium  \\ \hline
\textbf{Description} & When the user inputs a value for x a unreal non-numeric value and for y as a unreal non-numeric value, the Eternity Function shall display an error message\\ \hline
\textbf{Difficulty} & Nominal  \\ \hline
\textbf{Type} & Functional \\ 
\hline
\end{tabular} \\ \\ \\

\setlength{\tabcolsep}{18pt}
\renewcommand{\arraystretch}{1.5}
\begin{tabular}{ |p{2.2cm}|p{9.8cm}| }
\hline
\textbf{Identification} &  EF\_REQ\_6 \\ \hline 
\textbf{Version Number} & 1.0 \\ \hline 
\textbf{Owner} & Prashanthi Ramesh  \\ \hline
\textbf{Priority} & High  \\ \hline
\textbf{Risk} & Medium  \\ \hline
\textbf{Description} & When the user inputs a value for x as zero (0) and for y as zero (0), the Eternity Function shall display the result an error message\\ \hline
\textbf{Difficulty} & Nominal  \\ \hline
\textbf{Type} & Functional \\ 
\hline
\end{tabular} \\ \\ \\ 

\setlength{\tabcolsep}{18pt}
\renewcommand{\arraystretch}{1.5}
\begin{tabular}{ |p{2.2cm}|p{9.8cm}| }
\hline
\textbf{Identification} &  EF\_REQ\_7 \\ \hline 
\textbf{Version Number} & 1.0 \\ \hline 
\textbf{Owner} & Prashanthi Ramesh  \\ \hline
\textbf{Priority} & High  \\ \hline
\textbf{Risk} & Medium  \\ \hline
\textbf{Description} & When the user inputs a value for x as a real number and for y as real number less than zero (negative value), the Eternity Function shall display the result an error message\\ \hline
\textbf{Difficulty} & Nominal  \\ \hline
\textbf{Type} & Functional \\ 
\hline
\end{tabular} \\ \\ \\ 

\setlength{\tabcolsep}{18pt}
\renewcommand{\arraystretch}{1.5}
\begin{tabular}{ |p{2.2cm}|p{9.8cm}| }
\hline
\textbf{Identification} &  EF\_REQ\_8 \\ \hline 
\textbf{Version Number} & 1.0 \\ \hline 
\textbf{Owner} & Prashanthi Ramesh  \\ \hline
\textbf{Priority} & High  \\ \hline
\textbf{Risk} & Medium  \\ \hline
\textbf{Description} & When the user inputs a value for x as zero (0) and for y as real number, the Eternity Function shall display the result as 1\\ \hline
\textbf{Difficulty} & Nominal  \\ \hline
\textbf{Type} & Functional \\ 
\hline
\end{tabular} \\ \\ \\ 

\setlength{\tabcolsep}{18pt}
\renewcommand{\arraystretch}{1.5}
\begin{tabular}{ |p{2.2cm}|p{9.8cm}| }
\hline
\textbf{Identification} &  EF\_REQ\_9 \\ \hline 
\textbf{Version Number} & 1.0 \\ \hline 
\textbf{Owner} & Prashanthi Ramesh  \\ \hline
\textbf{Priority} & High  \\ \hline
\textbf{Risk} & Medium  \\ \hline
\textbf{Description} & When the user inputs a value for x as a real number less than zero (negative real number) and for y as an odd real number except zero (0), the Eternity Function shall display the result as negative value\\ \hline
\textbf{Difficulty} & Nominal  \\ \hline
\textbf{Type} & Functional \\ 
\hline
\end{tabular} \\ \\ \\ 

\setlength{\tabcolsep}{18pt}
\renewcommand{\arraystretch}{1.5}
\begin{tabular}{ |p{2.2cm}|p{9.8cm}| }
\hline
\textbf{Identification} &  EF\_REQ\_10 \\ \hline 
\textbf{Version Number} & 1.0 \\ \hline 
\textbf{Owner} & Prashanthi Ramesh  \\ \hline
\textbf{Priority} & High  \\ \hline
\textbf{Risk} & Medium  \\ \hline
\textbf{Description} & When the user inputs a value for x as a real number less than zero (negative real number) and for y as an even real number except zero (0), the Eternity Function shall display the result as positive value\\ \hline
\textbf{Difficulty} & Nominal  \\ \hline
\textbf{Type} & Functional \\ 
\hline
\end{tabular} \\ \\ \\ 

\setlength{\tabcolsep}{18pt}
\renewcommand{\arraystretch}{1.5}
\begin{tabular}{ |p{2.2cm}|p{9.8cm}| }
\hline
\textbf{Identification} &  EF\_REQ\_11 \\ \hline 
\textbf{Version Number} & 1.0 \\ \hline 
\textbf{Owner} & Prashanthi Ramesh  \\ \hline
\textbf{Priority} & Medium  \\ \hline
\textbf{Risk} & Low  \\ \hline
\textbf{Description} & The Eternity Function shall be available to users for use all the time\\ \hline
\textbf{Difficulty} & Nominal  \\ \hline
\textbf{Type} & Non-Functional \\ 
\hline
\end{tabular} \\ \\ \\ 

\setlength{\tabcolsep}{18pt}
\renewcommand{\arraystretch}{1.5}
\begin{tabular}{ |p{2.2cm}|p{9.8cm}| }
\hline
\textbf{Identification} &  EF\_REQ\_12 \\ \hline 
\textbf{Version Number} & 1.0 \\ \hline 
\textbf{Owner} & Prashanthi Ramesh  \\ \hline
\textbf{Priority} & Medium  \\ \hline
\textbf{Risk} & Low  \\ \hline
\textbf{Description} & When the Eternity Function is non-operational, the system shall present the user with notification informing them that the system is unavailable\\ \hline
\textbf{Raresent the usetionale} & Why  \\ \hline
\textbf{Difficulty} & Nominal  \\ \hline
\textbf{Type} & Non-Functional \\ 
\hline
\end{tabular} \\ \\ \\ 

\setlength{\tabcolsep}{18pt}
\renewcommand{\arraystretch}{1.5}
\begin{tabular}{ |p{2.2cm}|p{9.8cm}| }
\hline
\textbf{Identification} &  EF\_REQ\_13 \\ \hline 
\textbf{Version Number} & 1.0 \\ \hline 
\textbf{Owner} & Prashanthi Ramesh  \\ \hline
\textbf{Priority} & High  \\ \hline
\textbf{Risk} & Medium  \\ \hline
\textbf{Description} & The Eternity Function shall be able to handle entry value of double-precision floating-point.\\ \hline
\textbf{Difficulty} & Nominal  \\ \hline
\textbf{Type} & Non-Functional \\ 
\hline
\end{tabular} \\ \\ \\ 

\setlength{\tabcolsep}{18pt}
\renewcommand{\arraystretch}{1.5}
\begin{tabular}{ |p{2.2cm}|p{9.8cm}| }
\hline
\textbf{Identification} &  EF\_REQ\_14 \\ \hline 
\textbf{Version Number} & 1.0 \\ \hline 
\textbf{Owner} & Prashanthi Ramesh  \\ \hline
\textbf{Priority} & High  \\ \hline
\textbf{Risk} & Medium  \\ \hline
\textbf{Description} & The Eternity Function shall have a maximum response time of three seconds\\ \hline
\textbf{Difficulty} & Nominal  \\ \hline
\textbf{Type} & Non-Functional \\ 
\hline
\end{tabular} \\ \\ \\ 

\setlength{\tabcolsep}{18pt}
\renewcommand{\arraystretch}{1.5}
\begin{tabular}{ |p{2.2cm}|p{9.8cm}| }
\hline
\textbf{Identification} &  EF\_REQ\_15 \\ \hline 
\textbf{Version Number} & 1.0 \\ \hline 
\textbf{Owner} & Prashanthi Ramesh  \\ \hline
\textbf{Priority} & High  \\ \hline
\textbf{Risk} & Medium  \\ \hline
\textbf{Description} & The Eternity Function result value shall be accurate to fifteen decimal places\\ \hline
\textbf{Difficulty} & Nominal  \\ \hline
\textbf{Type} & Non-Functional \\ 
\hline
\end{tabular} \\ \\ \\ 

\setlength{\tabcolsep}{18pt}
\renewcommand{\arraystretch}{1.5}
\begin{tabular}{ |p{2.2cm}|p{9.8cm}| }
\hline
\textbf{Identification} &  EF\_REQ\_15 \\ \hline 
\textbf{Version Number} & 1.0 \\ \hline 
\textbf{Owner} & Prashanthi Ramesh  \\ \hline
\textbf{Priority} & Medium  \\ \hline
\textbf{Risk} & Low  \\ \hline
\textbf{Description} & The Eternity Function shall save the history of operations \\ \hline
\textbf{Difficulty} & Nominal  \\ \hline
\textbf{Type} & Functional \\ 
\hline
\end{tabular} \\ \\ \\ 

\setlength{\tabcolsep}{18pt}
\renewcommand{\arraystretch}{1.5}
\begin{tabular}{ |p{2.2cm}|p{9.8cm}| }
\hline
\textbf{Identification} &  EF\_REQ\_16 \\ \hline 
\textbf{Version Number} & 1.0 \\ \hline 
\textbf{Owner} & Prashanthi Ramesh  \\ \hline
\textbf{Priority} & Medium  \\ \hline
\textbf{Risk} & Low  \\ \hline
\textbf{Description} & The user shall view the history of operations \\ \hline
\textbf{Difficulty} & Nominal  \\ \hline
\textbf{Type} & Functional \\ 
\hline
\end{tabular} \\ \\ \\ 

\setlength{\tabcolsep}{18pt}
\renewcommand{\arraystretch}{1.5}
\begin{tabular}{ |p{2.2cm}|p{9.8cm}| }
\hline
\textbf{Identification} &  EF\_REQ\_17 \\ \hline 
\textbf{Version Number} & 1.0 \\ \hline 
\textbf{Owner} & Prashanthi Ramesh  \\ \hline
\textbf{Priority} & Low  \\ \hline
\textbf{Risk} & Low  \\ \hline
\textbf{Description} & The Eternity Function shall be easy to use by members of the public who may have at-least one hand free\\ \hline
\textbf{Difficulty} & Nominal  \\ \hline
\textbf{Type} & Usability \\
\hline
\end{tabular} \\ \\ \\ 

\setlength{\tabcolsep}{18pt}
\renewcommand{\arraystretch}{1.5}
\begin{tabular}{ |p{2.2cm}|p{9.8cm}| }
\hline
\textbf{Identification} &  EF\_REQ\_18 \\ \hline 
\textbf{Version Number} & 1.0 \\ \hline 
\textbf{Owner} & Prashanthi Ramesh  \\ \hline
\textbf{Priority} & Medium  \\ \hline
\textbf{Risk} & Medium  \\ \hline
\textbf{Description} & The Eternity Function shall be used by members of public without training \\ \hline
\textbf{Difficulty} & Nominal  \\ \hline
\textbf{Type} & Usability \\ 
\hline
\end{tabular} \\ \\ \\ 

\setlength{\tabcolsep}{18pt}
\renewcommand{\arraystretch}{1.5}
\begin{tabular}{ |p{2.2cm}|p{9.8cm}| }
\hline
\textbf{Identification} &  EF\_REQ\_19 \\ \hline 
\textbf{Version Number} & 1.0 \\ \hline 
\textbf{Owner} & Prashanthi Ramesh  \\ \hline
\textbf{Priority} & High  \\ \hline
\textbf{Risk} & Medium  \\ \hline
\textbf{Description} & A development programmer who has at least one year of experience supporting this software application shall be able to add a new product feature, including source code modifications and testing \\ \hline
\textbf{Difficulty} & Nominal  \\ \hline
\textbf{Type} & Non-functional\\ 
\hline
\end{tabular} \\ \\ \\ 

\setlength{\tabcolsep}{18pt}
\renewcommand{\arraystretch}{1.5}
\begin{tabular}{ |p{2.2cm}|p{9.8cm}| }
\hline
\textbf{Identification} &  EF\_REQ\_20 \\ \hline 
\textbf{Version Number} & 1.0 \\ \hline 
\textbf{Owner} & Prashanthi Ramesh  \\ \hline
\textbf{Priority} & Medium  \\ \hline
\textbf{Risk} & Medium  \\ \hline
\textbf{Description} & It shall be possible for the Eternity Function to be installed by a user who has no special expertise \\ \hline
\textbf{Difficulty} & Nominal  \\ \hline
\textbf{Type} & Non-functional\\ 
\hline
\end{tabular} \\ \\ \\ 

\setlength{\tabcolsep}{18pt}
\renewcommand{\arraystretch}{1.5}
\begin{tabular}{ |p{2.2cm}|p{9.8cm}| }
\hline
\textbf{Identification} &  EF\_REQ\_21 \\ \hline 
\textbf{Version Number} & 1.0 \\ \hline 
\textbf{Owner} & Prashanthi Ramesh  \\ \hline
\textbf{Priority} & High  \\ \hline
\textbf{Risk} & Medium  \\ \hline
\textbf{Description} & When a new version of the Eternity Function is released, it shall be possible to upgrade to it from any previous version \\ \hline
\textbf{Difficulty} & Nominal  \\ \hline
\textbf{Type} & Non-functional\\ 
\hline
\end{tabular} \\ \\ \\ 


\chapter{Problem 3}

\section{Pseudocode}

\textbf{\\ \\ Algorithm \ref{exp1}:} Taylor series is a representation of a function as an infinite sum of terms that are calculated from the values of the function's derivatives at a single point. 
\begin{equation} \label{evalpow}
x^y= e^{y\ln x}
\end{equation}
\ref{evalpow} evaluation of $x^y$. Here, e is a mathematical constant approximately equal to  2.71828\\ \\

\begin{equation} \label{extaylor}
e^x = 1 + x/1! + x^2/2! + x^3/3! + ...... 
\end{equation}
\ref{extaylor} express $e^x$ using Taylor Series\\ \\


\begin{equation} \label{altextaylore}
e^x = 1 + (x/1) (1 + (x/2) (1 + (x/3) (........) ) ) 
\end{equation}
\ref{altextaylore} The series \ref{extaylor} can be re-written as above\\ \\


\begin{equation} \label{logtaylor}
log(1+x) = x-x^2/2 + x^3/3- ... 
\end{equation}
\ref{logtaylor} express ln x using Taylor Series\\ \\

\setlength{\tabcolsep}{18pt}
\renewcommand{\arraystretch}{1.5}
\begin{tabular}{ |p{6cm}|p{6cm}| }
\hline
\textbf{Advantages} & \textbf{Disadvantages}\\ \hline 
Very useful for derivations
 & Successive terms get very complex and hard to derive\\
\hline
Can be used to get theoretical error bounds &Truncation error tends to grow rapidly away from expansion point\\
\hline
Object Reference Model parameters embedded as variables & fsdfds\\
\hline
Power series can be inverted to yield the inverse function & Almost always not as efficient as curve fitting or direct approximation\\
\hline
\end{tabular} \\ \\ \\ 

\begin{algorithm}
\caption{Exponentiation\cite{format} by Taylor Series}\label{exp1}
\begin{algorithmic}[1]
\Require $x \neq 0$ AND $y > 0$

\Function{logarithm}{$n$}\Comment{$algorithm for log(n)$}
\State $sum\gets 0$
\While{$n > 1$}
    \State $n\gets n/e$\Comment{e is a constant approximately equal to 2.71828}
    \State $y \gets y+1 $
\EndWhile \\
\Return $y$
\EndFunction

\Function{exponential}{$x$}\Comment{$algorithm for e^x$}
\State $sum\gets 1$
\State $n\gets 10$
\For{$i\gets n-1$, 1}
\State $sum\gets 1+ x * sum / i$
\EndFor \\
\Return $sum$
\EndFunction

\State $logx \gets $\Call{logarithm}{x}
\State $result \gets $\Call{exponential}{y*logx}
\end{algorithmic}
\end{algorithm}

\textbf{Algorithm \ref{exp2}:}
\begin{itemize}
    \item Approximation algorithms\cite{power} are efficient algorithms that find approximate solutions to NP-hard optimization problems with provable guarantees on the distance of the returned solution to the optimal one
    \item Neural network simulations often spend a large proportion of their time computing exponential functions. 
    \item Since the exponentiation routines of typical math libraries are rather slow, their replacement with a fast approximation can greatly reduce the overall computation time.
    \item An approximation is perfectly adequate for most neural computation purposes and can save much time.
    \item Exploiting the IEEE 754 floating-point representation to calculate $e^x$ \\ \\
\end{itemize}

\setlength{\tabcolsep}{18pt}
\renewcommand{\arraystretch}{1.5}
\begin{tabular}{ |p{6cm}|p{6cm}| }
\hline
\textbf{Advantages} & \textbf{Disadvantages}\\ \hline 
The calculation only requires 2 shifts, 1 multiplication, 2 addition, and 2 register operations.  & The algorithm efficiency gets worse as the values get larger and are not very accurate\\
\hline
The approximation function results are very fast. & Depends on IEEE 754 standard and can have compatibility issues\\
\hline

\end{tabular} \\ \\ \\ 

\begin{algorithm}
\caption{Approximation of the Exponential Function}\label{exp2}
\begin{algorithmic}[1]
\Require $x \neq 0$ AND $y > 0$
\State $x \gets >> 32$ \Comment{Calculate ln(x)}
\State $ln_x \gets x- 1072632447/ 1512775$
\State $exponent \gets ln_x * y$ \Comment{ln(x)*y}
\State $result \gets 1512775 * exponent +(1072693248 - 60801) (1072707600)$\Comment{$e^ln(x)*y$}
\State $result= result <<32$
\end{algorithmic}
\end{algorithm}



\appendix
\chapter{GitHub}
\section{Individual GitHub Link}
https://github.com/PrashanthiRamesh/SOEN-6011-Project-Calculator/

\section{Team GitHub Link}
https://github.com/niravjdn/SOEN-6011-Project/


\printbibliography

\printglossary

\end{document}